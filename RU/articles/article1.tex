\article{Основные положения}

\section{Гражданство}

\section{Высшие права и свободы}
\begin{enumerate}
    \item Данная Конституция признаёт следующие высшие права и свободы:
    \begin{itemize}
        \item право на жизнь;
         \item право на тело: право человека владеть собственным телом и распоряжаться им по своему усмотрению;
         \item право собственности: право владеть движимым и недвижимым имуществом, использовать его и распоряжаться им по своему усмотрению;
         \item право на справедливое отправление правосудия: право на рассмотрение иска о нарушении прав и свобод в установленном Конституцией порядке;
         \item свободу мысли;
         \item свободу совести: право человека формировать, иметь и менять свои собственные убеждения, в том числе исповедовать любую религию или не исповедовать никакой, как единолично, так и сообща с другими людьми;
    \end{itemize}
    \item Все граждане обладают равными и неотъемлимыми высшими правами и свободами.
    \item Осуществление высших прав и свобод человека и гражданина не должно нарушать права и свободы других лиц. 
    %NAP
    \item Лишение гражданина высших прав и свобод, а также их ограничение в любом порядке, кроме предусмотренного данной Конституцией, недопустимо. 
    
    
\end{enumerate}
\section{Контракт}
\begin{enumerate}
    
    
    
    
    \item Условия контракта не могут противоречить данной Конституции. Контракт, противоречащий Конституции, признаётся недействительным.
\end{enumerate}
