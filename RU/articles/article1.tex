\article{Основные положения}

\section{Субъекты Конституции} %Субъекты 

\section{Высшие права и свободы}
\begin{enumerate}
    \item Данная Конституция признаёт следующие высшие права и свободы:
    \begin{itemize}
        \item право на тело: право человека владеть собственным телом и единолично распоряжаться им по своему усмотрению;
        \item право собственности: право владеть движимым и недвижимым имуществом, использовать его и распоряжаться им по своему усмотрению;
        \item право на справедливое отправление правосудия: право на рассмотрение иска о нарушении прав и свобод в установленном Конституцией порядке;
        \item свободу мысли;
        \item свободу совести: право человека формировать, иметь и меняять свои собственные убеждения, в том числе исповедовать любую религию или не исповедовать никакой, как единолично, так и сообща с другими людьми;
    \end{itemize}
    \item Все граждане обладают равными, неотъемлимыми и неотчуждаемыми высшими правами и свободами.
    \item Объединения граждан, в случае, если эти объединения являются субъектами данной Конституции, обладают следущими высшими правами и свободами:
    \begin{itemize}
        \item правом собственности;
        \item правом на справедливое отправление правосудия;
        \item свободой организации.
    \end{itemize}
    \item Осуществление высших прав и свобод отдельных граждан и объединений не должно нарушать права и свободы других субъектов данной Конституции.
    %NAP
    \item Агрессией называется лишение субъекта Конституции высших прав и свобод, их ограничение в любом порядке, кроме предусмотренного данной Конституцией, а также угроза 
    \item , недопустимо.
    
\end{enumerate}
\section{Контракт}
\begin{enumerate}
    
    
    
    
    \item Условия контракта не могут противоречить данной Конституции. Контракт, противоречащий Конституции, признаётся недействительным.
\end{enumerate}
